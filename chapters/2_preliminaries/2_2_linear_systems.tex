\section{Linear Systems}
\label{sec:linear_systems}

\noindent A linear system is denoted by
\begin{equation}
\label{eqn:linear_system}
    Ax=b
\end{equation}
\noindent where $A \in \mathbb{R}^{n \times n}$ is called the \textit{coefficient matrix}, $b \in \mathbb{R}^{n}$ is the \textit{right-hand side} vector and $x \in \mathbb{R}^{n}$ is the \textit{vector of unknowns}. The task of solving a linear system is then reduced to finding a vector $x$ for a given input $A$ and $b$ such that Equation~\hyperref[eqn:linear_system]{\ref{eqn:linear_system}} is satisfied.

However, in many cases such an exact solution $x^*$ is unattainable on current computer architectures. Even if the algorithm used for solving the system is able to deliver the exact result in theory (and many algorithms, in fact, are), the numerical limits of floating-point calculation do still apply. Thus, the highest accuracy achievable is limited by the unit roundoff $u$ of the selected floating-point format. Furthermore, by taking the inevitable accumulation of rounding errors into account, it is most likely that the obtained result will not even be close to that level of accuracy. Despite that, the impact of rounding errors on the solution depends on its sensitivity to perturbations in the data. This is generally referred to as the \textit{conditioning} of a problem and consequently the error bounds are expressed in terms of the \textit{condition number} of a problem.

For linear systems, it can be shown that the conditioning is almost entirely dependent on the condition number of the matrix $A$ (see \cite{higham_accuracy_2002}) and the corresponding measurements are given below. The norm-wise condition number for a linear system can be calculated via:
\begin{equation}
\kappa_{p}(A) = \norm{A^{-1}}_p\norm{A}_p
\end{equation}

\noindent Regarding the component-wise condition number on the other hand, the following formulas have been established by \cite{skeel_scaling_1979}:
\begin{equation}
cond(A) = \norm{|A^{-1}||A|}_\infty \leq \kappa_{\infty}(A)
\end{equation}
\begin{equation}
cond(A,x) = \frac{\norm{|A^{-1}||A||x|}_\infty}{\norm{x}_\infty}
\end{equation}
\noindent where $|A| = |a_{i,j}|$ and $|x| = |x_{i}|$. These quantities represent how sensitive the solution of a linear system reacts to small norm-wise or component-wise perturbations and will be used according to these definitions in the remainder of this thesis.
