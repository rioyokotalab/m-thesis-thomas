\section{Dense Iterative Solvers}
\label{sec:iterative_solvers}

The goal of an iterative method is to generate a sequence of increasingly accurate approximate solutions $\hat{x}$ for a linear system $Ax=b$. In such a framework, the main goal it to replace the dense matrix $A$ by a matrix $M$ that is easier to solve (i.e. has a reduced complexity). This can either be achieved by a reduced size, or special characteristics that allow for solving in $\mathcal{O}(n^2)$ or $\mathcal{O}(n)$. Typically, in such a framework, the original matrix $A$ is then only involved in terms of matrix-vector multiplications, which can be calculated in order of $\mathcal{O}(n^2)$ as well. Therefore, as long as a desired accuracy can be achieved by a short sequence of approximate solutions, iterative methods can be considerably faster than direct ones, which require $\mathcal{O}(n^3)$. Depending on the general idea behind how this matrix $M$ is created, several different iterative methods can be distinguished. A summary of those approaches together with the most popular methods for each of them, will be provided in the following sections.

\import{2_3_iterative_solvers}{2_3_1_stationary_methods.tex}
\import{2_3_iterative_solvers}{2_3_2_subspace_methods.tex}
\import{2_3_iterative_solvers}{2_3_3_krylov_methods.tex}
\import{2_3_iterative_solvers}{2_3_4_preconditioners.tex}
\import{2_3_iterative_solvers}{2_3_5_preconditioned_methods.tex}

\newpage

