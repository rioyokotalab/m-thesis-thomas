\chapter{Iterative Refinement (IR)}
\label{chap:iterative_refinement}

The key idea of iterative refinement is to improve an initial inaccurate solution $\tilde{x}$ to a linear system $Ax=b$ by updating it with a correction term $d$, i.e. $\tilde{x}\leftarrow \tilde{x} + d$. This is achieved by computing the residual error $r=b-A\tilde{x}$ and solving the correction equation $Ad=r$. Those steps are then repeated until a desired accuracy is achieved. Initially developed by \cite{wilkinson_rounding_1963} to battle the influence of rounding errors when solving a linear system, the algorithm has been ported, modified and  re-evaluated under different (precision) environments (see \cite{moler_iterative_1967}, \cite{skeel_iterative_1980}, \cite{buttari_mixed_2007} and \cite{carson_accelerating_2018}). A general formulation of the algorithm that includes a variety of other ones as special cases is given by \cite{carson_accelerating_2018}, stating that each of the three main tasks (original solve/factorization, residual computation and solving for the correction equation) can be done at a different precision. In  their analysis, the algorithm contains the following four precisions that are denoted in terms of the unit roundoff:
\begin{itemize}
    \item $u$ is the working precision and used to store the original data $A$, $b$ and the approximate solution $\tilde{x}$
    \item $u_f$ is the precision in which the original solve/factorization is calculated
    \item $u_r$ is the precision used for calculating the residual
    \item $u_s$ is the achieved precision when solving the correction equation
\end{itemize}
In general, the following ordering applies to the precisions used:
\begin{equation}
    u_r \leq u \leq u_s \leq u_f
\end{equation}

\noindent Algorithm~\hyperref[alg:iterative_refinement]{\ref{alg:iterative_refinement}} (modified after \cite{carson_new_2017}) illustrates this general case of iterative refinement. Note that $u$, $u_f$ and $u_r$ are intended to be (possibly) different precisions, while $u_s$ is essentially a parameter describing how accurately the correction equation is solved. Depending on the solving technique employed, it will either take the value of $u_f$ or $u$ in the context of this research. Since the convergence of iterative refinement is restricted by the condition number of the matrix $A$, it is common to limit the maximum number of iterations by a value $i_{max}$ after which the algorithm is guaranteed to terminate, no matter whether convergence was achieved or not. Similarly, a tolerance value $\epsilon \geq 0$ can be specified for the approximate solution in order to detect early convergence if a desired accuracy has been achieved.  Note that the formulation of the stopping criterion in the algorithm can be formulated in terms of any desired norm $p$ and a detailed discussion of this topic can be found in \cite{demmel_error_2006}.

\begin{algorithm}[h]
  \caption{General iterative refinement in three precisions}
  \label{alg:iterative_refinement}
  \SetAlgoLined
  \KwIn{matrix $A \in \realx{n}$; vector $b \in \real$; iteration\# $k_{max}$; tolerance $\epsilon_{max}$}
  \KwOut{approximate solution $\iter[k]{x}$ to $Ax=b$}
  solve $A\iter[0]{x}=b$ in precision $u_f$ and store $x_0$ at precision $u$ \\
  \For{$k = 1$ \KwTo $k_{max}$} {
    compute $\iter[k]{r} = b-A\iter[k]{r}$ in precision $u_r$ and round $\iter[k]{r}$ to precision $u$ \\
    solve $A\iter[k]{d}=\iter[k]{r}$ in precision $u_s$ and store $\iter[k]{d}$ at precision $u$ \\
    $\iter[k+1]{x}=\iter[k]{x}+\iter[k]{d}$ in precision $u$}
  \% iteration has not converged
\end{algorithm}

\import{}{4_1_mixed_precision_ir.tex}
\import{}{4_2_low_rank_ir.tex}
\import{}{4_3_combined_ir.tex}