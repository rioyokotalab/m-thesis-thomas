\section{Structure of this Thesis}
\label{sec:structure}

This thesis is intended to serve as a self-contained introduction to iterative refinement and low-rank approximations, providing the reader with the necessary foundations used to build the proposed algorithm. General mathematical notations and definitions that will be used throughout this work are detailed in Chapter~\hyperref[chap:preliminaries]{\ref{chap:preliminaries}}, establishing the basic concepts of linear systems and matrix rank.

Chapter~\hyperref[chap:solvers]{\ref{chap:solvers}} contains an overview of solving techniques for dense linear systems. Starting from dense linear systems in Section~\hyperref[sec:direct_solvers]{\ref{sec:direct_solvers}} it will gradually introduce more involved iterative concepts in an incremental fashion from basic iterative methods up to current state-of-the-art Krylov subspace based approaches. Special consideration is given to the topic of preconditioners in Section~\hyperref[sec:preconditioners]{\ref{sec:preconditioners}} as they have an important role in the proposed algorithm.

Chapter~\hyperref[chap:hierarchical_matrices]{\ref{chap:hierarchical_matrices}} is then dedicated to the topic of hierarchical matrices, introducing the most common formats as well as their benefits. After establishing the basics in Section~\hyperref[sec:matrix_partition]{\ref{sec:matrix_partition}}, does not only discuss different types (such as block low-rank and hierarchical semi-separable, to name a few), but also explains the concept of shared bases and its implications. Implications of the different formats on typical matrix operations are then detailed in the last section of this chapter. 

The existing research in the field of iterative refinement is summarized in Chapter~\hyperref[chap:iterative_refinement]{\ref{chap:iterative_refinement}} with a focus on proposed mixed precision methods. The theoretical benefits of the combination with hierarchical low-rank matrices are then explored in Section~\hyperref[sec:sec:low_rank_ir]{\ref{sec:low_rank_ir}} \& Section~\hyperref[sec:combinded_ir]{\ref{sec:combinded_ir}} before detailing the results obtained via numerical experiments in Chapter~\hyperref[chap:experiments]{\ref{chap:experiments}}. Error bounds and accuracy criteria are the main focus in this discussion but a performance evaluation will be provided as well. This is followed by a summary of the main findings and implications for future work in Chapter~\hyperref[chap:conclusion]{\ref{chap:conclusion}}, that serves as a conclusion to this thesis.