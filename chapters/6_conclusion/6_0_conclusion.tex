\chapter{Conclusion \& Future Work}
\label{chap:conclusion}

A high performance implementation of iterative refinement in three precisions was investigated in this thesis. In principle, the results presented by \cite{carson_accelerating_2018}, showing that it is possible to obtain solutions accurate to working precision even if the factorization is done in lower precision, could be confirmed. However, for a setting where only single and double precision arithmetics are implemented in hardware, the general cost of software emulated high precision calculations remains high, limiting the obtained performance. With regards to the efficiency of the algorithm, this is a major drawback of the approach. Nonetheless, the increasing support for half precision arithmetics in CPUs and GPUs opens up a large field for future experiments. In such a setting, the overhead of the high precision computations could be significantly reduced, while still guaranteeing result accurate to single precision. Extending and testing the implementation in such a setting is thus the natural next step for further research.

In addition to the general results about iterative refinement, the focus of this research was on the combination with hierarchical low-rank approximations, aiming to reduce the overall runtime complexity of the algorithm. Instead of reducing the factorization cost by using lower precision calculations, the original matrix is approximated by a hierarchy of low-rank representations. The reduced accuracy of such a matrix format enables a faster decomposition algorithm, that can be used to obtain approximate LU factors. Using those to replace the low precision (but exact) factors in the iterative refinement process showed to produce similar results at a reduced complexity. In fact, convergence could be observed even if the approximation accuracy was reduced down to $10^{-2}$ as long as the factorization was done in working precision. This applied to the whole set of test matrices with condition numbers $\kappa_\infty(A) \leq 10^8$. 

Due to the reduced cost of the factorization, the overall complexity of the algorithm is now $\mathcal{O}(n^2)$ (instead of $\mathcal{O}(n^3)$ for general iterative refinement), which is supported by the numerical experiments. However, it needs to be mentioned that due to the highly optimized kernels being available for dense LU decomposition, the proposed variant is only expected to be efficient when the problem size is large. For matrices with a size of up to $n = 2048$, the average runtime measured in the experiments remained above the time needed by a direct solver. The efficiency is further reduced by the frequent conversion of data-types (i.e. precisions) and optimizing the memory management is another field for future investigations. Even though, it is expected that solutions to large linear systems can be achieved significantly faster with the proposed method, but more time is needed for experimental verification.

In order to obtain an even greater speed-up of the factorization, low-rank iterative refinement based on GMRES seems to be fully compatible with a three precision approach. In such a setting, both the construction of the hierarchical matrix as well as the decomposition can be done at lower-precision (e.g. single or half), reducing the associated arithmetic cost. Again the results obtained in the numerical experiments are identical to \cite{carson_accelerating_2018}, even though a hierarchical LU is employed. In other words, for ill-conditioned matrices and a lower precision LU, it is possible to maintain the same error bounds if GMRES is used to calculate the correction term. However, the resulting higher number of inner GNRES iterations incurs a higher overhead, leading to an increase in runtime. As of yet, it remains unclear whether or not this is advantageous to a hierarchical factorization in higher-precision, since GMRES becomes the new bottleneck in such a case.

Currently, the advantages of GMRES-based IR are still not fully understood, even for the  non-hierarchical case and further research in this field is necessary. Especially clear indications on when GMRES-based IR becomes more beneficial compared to (low precision) LU-based IR are missing. As the application of the algorithm to high-performance applications would inevitably need to make an informed decision on which variant to choose, such information would be invaluable. Therefore, any future research in this area needs to include a more fine-grained performance analysis, providing insights into such details.

In summary, this thesis demonstrated that the error bound of mixed precision iterative refinement can be directly transferred to low-rank approximations. Given that the hierarchical representation is accurate to the lower-precision, the same forward and backward errors can be guaranteed, while reducing the factorization cost down to $\mathcal{O}(n\;log(n))$. This is a significant improvement over standard iterative refinement as it reduces the overall complexity of the algorithm down to $\mathcal{O}(n^2)$. If desired, it can even be combined with a three precision approach at the cost of additional iterations if the condition number of the matrix is large.


\chapter{Acknowledgements}
\label{chap:acknowledgements}

I would like to take this opportunity to express my gratitude towards Professor Rio Yokota, who served as a supervisor for this thesis. My research would not have been possible without his guidance, providing my with support whenever I encountered trouble along the way. Apart from offering counsel he also assured that I had access to all the computing resources needed, providing me with the perfect environment to conduct my research. I would also like to thank my colleagues Peter Spalthoff, Sameer Deshmukh, Qiancing Ma and Ridwan Apriansyah who together helped to build the hierarchical library I used for my experiments. Additionally, they offered valuable insights and constructive discussions on the topic, that helped my understanding greatly. Additionally I would like to express my gratitude towards the other members of Rio Yokota laboratory who offered invaluable support while I was getting adjusted to my life in Japan. I am looking forward to deepen these bonds even further in the next three years as a PhD studend among these wonderful people.

I also want to thank my parents and family who, even if they were not happy with my decision to come to Japan, nonetheless did their best to support me all the way along. Although the distance no has grown farther then ever, they are the persons mainly responsible for shaping my life and I am grateful for the wonderful experiences we shared together. It is my firm belief that their guidance helped me to become the person I am today and I hope I will have the opportunity to repay this debt in the future.